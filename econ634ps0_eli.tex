\documentclass[12pt]{article}%
\usepackage[paper=portrait,pagesize]{typearea}
\usepackage{amssymb}
\usepackage{amsfonts}
\usepackage{amsmath}
\usepackage{hyperref}
\usepackage{lscape}
\usepackage{comment}
\usepackage[flushleft]{threeparttable}
\usepackage{float}
\usepackage[nohead]{geometry}
\usepackage[singlespacing]{setspace}
\usepackage[paper=portrait,pagesize]{typearea}
\usepackage{amssymb}
\usepackage{amsfonts}
\usepackage{multicol}
\usepackage{amsmath}
\usepackage{hyperref}
\usepackage[nameinlink,noabbrev]{cleveref}
\usepackage{lscape}
\usepackage{float}
\usepackage[nohead]{geometry}
\usepackage[singlespacing]{setspace}
\usepackage[bottom]{footmisc}
\usepackage{indentfirst}
\usepackage{endnotes}
\usepackage{graphicx}%
\usepackage{afterpage}
\usepackage{subfig}
\usepackage{rotating}
\newcommand\tab[1][1cm]{\hspace*{#1}}
\DeclareMathOperator*{\Max}{Max}
\newcommand\numberthis{\addtocounter{equation}{1}\tag{\theequation}}
%\usepackage[backend=biber,style=alphabetic,sorting=ynt]{biblatex}
%\addbibresource{bibliocopulas.bib}
\usepackage[round,sort,comma,authoryear]{natbib}
\setcounter{MaxMatrixCols}{30}
\newtheorem{theorem}{Theorem}
\newtheorem{acknowledgement}{Acknowledgement}
\newtheorem{algorithm}[theorem]{Algorithm}
\newtheorem{axiom}[theorem]{Axiom}
\newtheorem{case}[theorem]{Case}
\newtheorem{claim}[theorem]{Claim}
\newtheorem{conclusion}[theorem]{Conclusion}
\newtheorem{condition}[theorem]{Condition}
\newtheorem{conjecture}[theorem]{Conjecture}
\newtheorem{corollary}[theorem]{Corollary}
\newtheorem{criterion}[theorem]{Criterion}
\newtheorem{definition}[theorem]{Definition}
\newtheorem{example}[theorem]{Example}
\newtheorem{exercise}[theorem]{Exercise}
\newtheorem{lemma}[theorem]{Lemma}
\newtheorem{notation}[theorem]{Notation}
\newtheorem{problem}[theorem]{Problem}
\newtheorem{proposition}{Proposition}
\newtheorem{remark}[theorem]{Remark}
\newtheorem{solution}[theorem]{Solution}
\newtheorem{summary}[theorem]{Summary}
\newenvironment{proof}[1][Proof]{\noindent\textbf{#1.} }{\ \rule{0.5em}{0.5em}}
\newcommand{\pd}[2]{\frac{\partial#1}{\partial#2}}
\makeatletter
\def\@biblabel#1{\hspace*{-\labelsep}}
\makeatother
\geometry{left=1in,right=1in,top=1.00in,bottom=1.0in}
%\renewcommand*\abstractname{Summary}

\begin{document}

\title{Fall 2019 - ECON 634 - Advance Macroeconomics - Problem Set 0}
\author{Elisa Taveras Pena\footnote{E-mail address: \href{mailto:etavera2@binghamton.edu}{etavera2@binghamton.edu}  }\\
Binghamton University}
\maketitle

\sloppy%avoids the breakage of words at the end of lines, by adjusting spaces between words inside the lines

\onehalfspacing



%\footnote{\underline{Acknowledgement}: The collection of data used in this study was partly supported by the National Institutes of Health under grant number R01 HD069609 and R01 AG040213, and the National Science Foundation under award numbers SES 1157698 and 1623684. }

%\strut

%\textbf{Keywords:} Urban growth control; Land use
%regulation; Development; Regional Inequality; Labor supply.

\strut

% \textbf{JEL Classification Numbers:} R14, H0, XY.

%\pagebreak%breaks to the next page
%\doublespacing %makes space between lines to be double, use singlespacing for space 1
\onehalfspacing

\end{document}
